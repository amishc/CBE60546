% Created 2021-09-28 Tue 20:29
% Intended LaTeX compiler: pdflatex
\documentclass[11pt]{article}
\usepackage[utf8]{inputenc}
\usepackage{lmodern}
\usepackage[T1]{fontenc}
\usepackage[top=1in, bottom=1.in, left=1in, right=1in]{geometry}
\usepackage{graphicx}
\usepackage{longtable}
\usepackage{float}
\usepackage{wrapfig}
\usepackage{rotating}
\usepackage[normalem]{ulem}
\usepackage{amsmath}
\usepackage{textcomp}
\usepackage{marvosym}
\usepackage{wasysym}
\usepackage{amssymb}
\usepackage{amsmath}
\usepackage[theorems, skins]{tcolorbox}
\usepackage[version=3]{mhchem}
\usepackage[numbers,super,sort&compress]{natbib}
\usepackage{natmove}
\usepackage{url}
\usepackage[cache=false]{minted}
\usepackage[strings]{underscore}
\usepackage[linktocpage,pdfstartview=FitH,colorlinks,
linkcolor=blue,anchorcolor=blue,
citecolor=blue,filecolor=blue,menucolor=blue,urlcolor=blue]{hyperref}
\usepackage{attachfile}
\usepackage{setspace}
\usepackage[left=1in, right=1in, top=1in, bottom=1in, nohead]{geometry}
\geometry{margin=1.0in}
\usepackage{hyperref}
\usepackage{amsmath}
\usepackage{graphicx}
\usepackage{epstopdf}
\usepackage{fancyhdr}
\pagestyle{fancy}
\fancyhf{}
\usepackage[labelfont=bf]{caption}
\usepackage{setspace}
\setlength{\headheight}{10.2pt}
\setlength{\headsep}{20pt}
\renewcommand{\headrulewidth}{0.5pt}
\renewcommand{\footrulewidth}{0.5pt}
\lfoot{\today}
\cfoot{\copyright\ 2021 W.\ F.\ Schneider}
\rfoot{\thepage}
\chead{\bf{Advanced Chemical Reaction Engineering (CBE 60546)\vspace{12pt}}}
\lhead{\bf{Homework 4}}
\rhead{\bf{Due October 1, 2021}}
\usepackage{titlesec}
\titlespacing*{\section}
{0pt}{0.6\baselineskip}{0.2\baselineskip}
\title{University of Notre Dame\\Advanced Chemical Engineering Thermodynamics\\(CBE 60553)}
\author{Prof. William F.\ Schneider}
\usepackage{siunitx}
\usepackage[version=3]{mhchem}
\def\dbar{{\mathchar'26\mkern-12mu d}}
\setcounter{secnumdepth}{3}
\author{William F. Schneider}
\date{\today}
\title{CBE 60546 Homework}
\begin{document}

\begin{OPTIONS}
\end{OPTIONS}

\noindent \textbf{Carefully and neatly document your answers.  You may use a mathematical solver like Jupyter/iPython. Use plotting software for all plots.}

\section{It's elementary, really\ldots{}}
\label{sec:org9095c72}
\subsection{Which of the following reactions can be assumed to be elementary?  If it is elementary, indicate the molecularity. Briefly justify your answers.}
\label{sec:org63679cc}

\begin{center}
\begin{tabular}{c}
\ce{ O3 (g) -> O2 (g) + O (g) }\\
\\
\ce{ 2 H2 (g) + O2(g) -> 2 H2O (g) }\\
\\
\ce{2 NO (g) + O2 (g) -> 2 NO2 (g) }\\
\\
\ce{H* + I2 -> HI + I*}\\
\\
2 \ce{Pt(111) + H2 (g) ->  2 Pt(111)-H}\\
\end{tabular}
\end{center}

\noindent In the last example, \ce{Pt(111)} indicates the surface of a Pt particle.

\section{The path less traveled}
\label{sec:org992f3e8}
Chlorine monoxide, \ce{ClO}, exhibits three different self-reaction channels:

\begin{center}
\begin{tabular}{lcc}
\hline
 & \(A\) (\si{\liter\per\mole\per\second}) & \(E_a\) (\si{\kilo\joule\per\mole})\\
\hline
\ce{ClO* + ClO* -> Cl2 + O2} & \SI{6.08e8} & 13.2\\
 &  & \\
\ce{ClO* + ClO* -> Cl* + ClOO} & \SI{1.79e10} & 20.4\\
 &  & \\
\ce{ClO* + ClO* -> Cl* + OClO} & \SI{2.11e8} & 11.4\\
\hline
\end{tabular}
\end{center}


\subsection{Make Arrhenius plots of the three reactions from 150 to \SI{500}{K}.  Which reaction dominates at low temperature?  At high temperature?}
\label{sec:org025a840}

\subsection{Can collision theory account for the variations in rate constants amongst the three reactions?  Why or why not?}
\label{sec:org872d1eb}

\subsection{The rate constant for a gas-phase bimolecular reaction can be written within transition state theory (using an isobaric standard state) as shown below. Use the definition of \(E_a\) to derive relationships between the Arrhenius parameters and the standard activation enthalpy, \(\Delta H^{\circ\ddagger}\) and standard activation entropy, \(\Delta S^{\circ\ddagger}\), at \SI{1}{bar} standard state.}
\label{sec:orgb11d510}

\[ k = \frac{k_B T}{h} \left ( \frac{RT}{P^\circ} \right ) e^{-\Delta G^{\circ(T)\ddagger}/k_B T} \]

\subsection{Calculate \(\Delta H^{\circ\ddagger}\) and \(\Delta S^{\circ\ddagger}\) for each of the three reactions.}
\label{sec:org6c72399}

\subsection{Using chemical intuition and the calculated \(\Delta S^{\circ\ddagger}\) as guides, draw candidate transition state structures for each of the three reactions.  Include arrows to show motion along the reaction coordinates.}
\label{sec:org173f392}

\subsection{Following is some thermodynamic data. Use it to sketch/draw/plot a potential enthalpy surface for the three reactions and a potential free energy surface for the three reactions at \SI{298}{K}.}
\label{sec:orgdb32d31}

\begin{center}
\begin{tabular}{lcc}
\hline
 & \(\Delta H_f^\circ(\SI{298}{K})\) & \(\Delta S^\circ(\SI{298}{K})\)\\
 & (\si{\kilo\joule\per\mole}) & (\si{\joule\per\mole\per\kelvin})\\
\hline
\ce{ClO*} & 101.22 & 226.65\\
\ce{OClO} & 104.60 & 257.22\\
\ce{ClO2} & 98.0 & 269.32\\
\ce{Cl} & 121.3 & 165.19\\
\ce{Cl2} & 0 & 223.08\\
\ce{O2} & 0 & 205.15\\
\hline
\end{tabular}
\end{center}

\section{What do I get in compensation?}
\label{sec:org0d22662}
Following are some observed data for the dehydrogenation of \emph{iso}-propanol, \ce{CH3CH(OH)CH3 -> CH3C(O)CH3 + H2}, over a series of catalysts:

\begin{center}
\begin{tabular}{cc}
\hline
\(A\) (\si{\per\second}) & \(E_a\) (\si{\kilo\joule\per\mole})\\
\hline
\SI{4.3e12}{} & 172\\
\SI{2.3e11}{} & 159\\
\SI{2.2e10}{} & 146\\
\SI{1.6e9}{} & 134\\
\hline
\end{tabular}
\end{center}

\subsection{What can you say about the relationship between \(\Delta S^{\circ\ddagger}\) and \(\Delta Hp^{\circ\ddagger}\)?}
\label{sec:org8579856}
\section{The radical path}
\label{sec:orgda06d2a}
At high temperature ethylene can be hydrogenated to ethane.  The proposed mechanism has four steps, all of which are presumed to be elementary and essentially irreversible under realistic conditions.

\begin{center}
\begin{tabular}{c}
\ce{ C2H4 (g)+ H2 (g)  ->[k_1] C2H5* (g) + H* (g) }\\
\\
\ce{H* (g) + C2H4 (g) ->[k_2]  C2H5* (g) }\\
\\
\ce{C2H5* (g) + H2 (g) ->[k_3] C2H6(g) (g) + H* (g) }\\
\\
\ce{C2H5* (g) + H* (g) ->[k_4] C2H6 (g)}\\
\end{tabular}
\end{center}


\subsection{Identify the reactants, products, and intermediates, and state whether the mechanism is open or closed.}
\label{sec:org10d1676}

\subsection{Why is it reasonable to assume that the first reaction is irreversible (that is, that the forward reaction rate is much greater than the reverse)?  What about the second reaction?}
\label{sec:org7fa7adb}

\subsection{Based on the mechanism above, write an expression for the rate of disappearance of ethylene.}
\label{sec:org4fcd4a4}

\subsection{Apply the quasi-steady-state approximation separately to H atoms and ethyl radicals.  Use the results to derive expressions for the concentrations of each in terms of only reactants and products.}
\label{sec:org03a63e4}

\subsection{Combine your answers to  obtain an expression for the rate of disappearance of ethylene that involves only reactants and products.  What is the apparent reaction order with respect to \ce{H2}?  To \ce{C2H4}?}
\label{sec:orgbbee3f2}

\subsection{What is the apparent rate constant?  Do you expect it to exhibit Arrhenius behavior in general?}
\label{sec:org170d657}

\subsection{Ethane dissociation \ce{C2H6 -> 2 CH3} is a key step in the initiation of gas-phase ethane reactions. If the reaction is observed in a diluent, say \ce{N2}, the rate is found to be a function of total pressure and to reach a limiting value at high pressure. Why? Can you propose a simple model for this behavior?}
\label{sec:orgd0366a0}
\end{document}