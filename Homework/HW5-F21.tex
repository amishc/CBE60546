% Created 2021-10-08 Fri 14:41
% Intended LaTeX compiler: pdflatex
\documentclass[11pt]{article}
\usepackage[utf8]{inputenc}
\usepackage{lmodern}
\usepackage[T1]{fontenc}
\usepackage[top=1in, bottom=1.in, left=1in, right=1in]{geometry}
\usepackage{graphicx}
\usepackage{longtable}
\usepackage{float}
\usepackage{wrapfig}
\usepackage{rotating}
\usepackage[normalem]{ulem}
\usepackage{amsmath}
\usepackage{textcomp}
\usepackage{marvosym}
\usepackage{wasysym}
\usepackage{amssymb}
\usepackage{amsmath}
\usepackage[theorems, skins]{tcolorbox}
\usepackage[version=3]{mhchem}
\usepackage[numbers,super,sort&compress]{natbib}
\usepackage{natmove}
\usepackage{url}
\usepackage[cache=false]{minted}
\usepackage[strings]{underscore}
\usepackage[linktocpage,pdfstartview=FitH,colorlinks,
linkcolor=blue,anchorcolor=blue,
citecolor=blue,filecolor=blue,menucolor=blue,urlcolor=blue]{hyperref}
\usepackage{attachfile}
\usepackage{setspace}
\usepackage[left=1in, right=1in, top=1in, bottom=1in, nohead]{geometry}
\geometry{margin=1.0in}
\usepackage{hyperref}
\usepackage{amsmath}
\usepackage{graphicx}
\usepackage{epstopdf}
\usepackage{fancyhdr}
\pagestyle{fancy}
\fancyhf{}
\usepackage[labelfont=bf]{caption}
\usepackage{setspace}
\setlength{\headheight}{10.2pt}
\setlength{\headsep}{20pt}
\renewcommand{\headrulewidth}{0.5pt}
\renewcommand{\footrulewidth}{0.5pt}
\lfoot{\today}
\cfoot{\copyright\ 2021 W.\ F.\ Schneider}
\rfoot{\thepage}
\chead{\bf{Advanced Chemical Reaction Engineering (CBE 60546)\vspace{12pt}}}
\lhead{\bf{Homework 5}}
\rhead{\bf{Due October 12, 2021}}
\usepackage{titlesec}
\titlespacing*{\section}
{0pt}{0.6\baselineskip}{0.2\baselineskip}
\title{University of Notre Dame\\Advanced Chemical Engineering Thermodynamics\\(CBE 60553)}
\author{Prof. William F.\ Schneider}
\usepackage{siunitx}
\usepackage[version=3]{mhchem}
\def\dbar{{\mathchar'26\mkern-12mu d}}
\setcounter{secnumdepth}{3}
\author{William F. Schneider}
\date{\today}
\title{CBE 60546 Homework}
\begin{document}

\begin{OPTIONS}
\end{OPTIONS}

\noindent \textbf{Carefully and neatly document your answers.  You may use a mathematical solver like Jupyter/iPython. Use plotting software for all plots.}

\section{Looking only at the surface}
\label{sec:org953b4d6}
One method of determining the surface area of Pt catalysts is by surface titration with \ce{H2}. \ce{H2} is thought to adsorb dissociatively, one H atom per surface Pt:

\[ \ce{H2 + 2\ast{} <=> 2 H\ast{}} \]
\subsection{Derive the Langmuir isotherm for dissociative adsorption of \ce{H2}.}
\label{sec:org0aaede4}

\subsection{The data below were obtained for isothermal \ce{H2} chemisorption on a 3\% (w/w) Pt catalyst supported on \(\gamma\)-alumina (nominally \ce{Al2O3}). Does the data conform to a dissociative Langmuir adsorption model?}
\label{sec:org6302f21}

\begin{center}
\begin{tabular}{cc}
\hline
Pressure (bar) & mass \ce{H2}/catalyst (\si{\micro\gram\per\gram})\\
\hline
0.0759 & 1.30\\
0.152 & 1.82\\
0.228 & 2.30\\
0.304 & 2.44\\
0.378 & 2.55\\
1.518 & 4.51\\
1.896 & 4.59\\
2.276 & 4.95\\
2.662 & 5.21\\
3.035 & 5.35\\
3.408 & 5.50\\
3.781 & 5.65\\
\hline
\end{tabular}
\end{center}

\section{Too hot to stick}
\label{sec:org701df5e}
Temperature-programmed desorption (TPD) is a common way to explore the kinetics of desorption of gases from solid surfaces. Bray et al explored models for the associative desorption of \ce{O2} from a Pt(111) surface, a process that has an activation barrier that depends on the coverage (\url{http://dx.doi.org/10.1016/j.susc.2013.12.005}):

\[ \ce{2 O\ast{} ->[k(\theta)] O2(g) + 2 \ast{}} \]

\noindent Bray used a desorption prefactor of about \SI{1e14}{\per\second} and found the desorption activation energy to depend approximately linearly on oxygen coverage, from \SI{2.5}{eV} at zero coverage to \SI{2.1}{eV} at \SI{0.3}{ML}.  

\subsection{Plot (on one graph) the \ce{O2} desorption rate vs temperature, starting from 0.0073, 0.093, 0.164, 0.194, and 0.5 ML O and using a temperature ramp rate of \SI{8}{\kelvin\per\second}.}
\label{sec:orga0652c4}
\section{Langmuir, Hinshelwood, and company}
\label{sec:org85fe6d0}
Bob Davis and students (\url{https://doi.org/10.1006/jcat.1999.2780}) studied the reduction of nitrous oxide by carbon monoxide over a ceria-supported Rh catalyst.  Following are rate data obtained at \SI{543}{K}.

\begin{center}
\begin{tabular}{ccc}
\hline
\(P_{\ce{CO}}\) (torr) & \(P_{\ce{N2O}}\) (torr) & TOF (\si{\per\second})\\
\hline
30.4 & 7.6 & 0.005\\
30.4 & 15.2 & 0.0091\\
30.4 & 30.4 & 0.018\\
30.4 & 45.6 & 0.023\\
30.4 & 76 & 0.036\\
7.6 & 30.4 & 0.039\\
15.2 & 30.4 & 0.024\\
45.6 & 30.4 & 0.012\\
76 & 30.4 & 0.0078\\
\hline
\end{tabular}
\end{center}

\subsection{What is the apparent rate order with respect to \ce{CO}? With respect to \ce{N2O}?}
\label{sec:orgc069e64}

\subsection{Following is a candidate mechanism. Derive a Langmuir-Hinshelwood rate expression assuming the first two reactions are quasi-equilibrated and the third reaction is rate-limiting.}
\label{sec:org8e54c58}

\[ \ce{N2O + \ast{} <=>[k_1][k_{-1}] N2O^\ast} \]
\[ \ce{CO + \ast{} <=>[k_2][k_{-2}] CO^\ast} \]
\[ \ce{N2O^\ast{} ->[k_3] N2 + O^\ast} \]
\[ \ce{CO^\ast{} + O^\ast{} ->[k_4] CO2 } \]

\subsection{Use the observed data and regression to estimate the rate parameters. (\emph{Hint}: Linearize to estimate the rate parameters, and use these as guesses for non-linear regression.)}
\label{sec:org8f7fac8}

\subsection{The reaction rate is observed to be Arrhenius over the temperature range 500 to \SI{550}{K} with apparent activation energy \SI{140}{\kilo\joule\per\mole}. Can you rationalize this result with your model?}
\label{sec:org9eb99a6}

\section{Peak performance}
\label{sec:org2eebe53}
Ammonia synthesis (\ce{N2 + 3 H2 <=> 2 NH3}) is among the most important heterogeneous catalytic reactions and has been studied extensively. Mehta (\url{https://doi.org/10.1021/acscatal.0c00684}) following Grabow write a simple lumped model for the mechanism over a metal catalyst:

\begin{center}
\begin{tabular}{lcccc}
\hline
 & \(E_a\) & \(\Delta S^{\circ\ddagger}\) & \(\Delta E\) & \(\Delta S^\circ\)\\
 & (eV) & (J/mol K) & (eV) & (J/mol K)\\
\hline
\(\ce{N2 + 2\ast{} <=> 2 N^\ast{} }\) & \(1.57 E_N + 1.56\) & \(-216.9\) & \(2 E_N\) & \(-216.9\)\\
\(\ce{N^\ast{} + 3/2 H2 <=> NH3 + \ast{}}\) & \(-0.39 E_N + 1.24\) & \(-5.6\) & \(-0.55-E_N\) & \(-5.6\)\\
\hline
\end{tabular}
\end{center}

\noindent The activation energies for both steps are related through a Br\o{}nsted-Evans-Polanyi relationship to the binding energy of N, \(E_N\). Those relationships are shown above, alone with some entropy data at \SI{1}{bar} standard state. Note \(k_B = \SI{8.6173e-5}{eV/K}\).

\subsection{Determine and plot the log rate per site (turnover frequency) at \SI{700}{K}, \SI{100}{bar}, and a stoichiometric mixture of \ce{N2} and \ce{H2} at 1\% conversion, as a function of \(E_N\) from \(-1.3\) to \(0\) eV.  (\emph{Hint}: First apply quasi-steady-state approximation to \(\theta_N\), then back solve for rate.)}
\label{sec:org43038e6}

\subsection{Typical catalysis Fe and Ru have binding energies \(-1.2\)  and \(-0.5\) eV.  Can you see why these are useful catalysts?}
\label{sec:org82062e8}
\end{document}